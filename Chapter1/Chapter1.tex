%=== Chapter One ===
\chapter{Introduction}

\ifpdf
    \graphicspath{{Chapter1/Chapter1Figs/PNG/}{Chapter1/Chapter1/PDF/}{Chapter1/Chapter1Figs/}}
\else
    \graphicspath{{Chapter1/Chapter1Figs/EPS/}{Chapter1/Chapter1/}}
\fi

\section{Introduction}

Defining and distinguishing phases of matter has been a continuing effort by physicists for many years. 

\section{Free Fermion Systems}

\subsection{Dirac Fermions}
The basic constituent of fermionic systems is the Dirac fermion. In the language of quantum field theory, a Dirac fermion can be represented by a second quantised field operator $a$, called the annihilation operator, and its conjugate partner $a^{\da}_{i}$, called the creation operator. They obey the anticommutation relations $\{a,a^{\da}\}=1$ and $\{a^{(\da)},a^{(\da)}\}=0$. These operators act on a number state, also known as a Fock state, in the following way
\begin{align}
    &a\ket{0}=0 &a^{\da}\ket{0}=\ket{1}\nn\\
    &a\ket{1}=\ket{0} &a^{\da}\ket{1}=0
\end{align}

\noi which follow from the commutation relations. A general state of a system consisting of a single Dirac fermion can be written in the form 
\begin{equation}
    \ket{\psi}=\alpha_{0}\ket{0}+\alpha_{1}\ket{1},
\end{equation}

\noi where $\alpha_{0},\alpha_{1}\in\mathbb{C}$.\\

Given $N\in\mathbb{N}^{+}$ Dirac fermions, they can be represented by a set of second quantised fermionic field operators $\{a_{i}\}$ and their conjugate partners $\{a_{i}^{\da}\}$, where $i=1,...,N$. They obey the following commutation relations
\begin{align}
    &\{a_{i},a_{j}^{\da}\}=\delta_{ij} &\{a_{i}^{(\da)},a_{j}^{(\da)}\}
\end{align}
    
\noi where $\delta_{ij}$ Kronercker delta function. These operators act on a tensor product of Fock states. A general state of the fermionic system, $\ket{\psi}$ can be written as
\begin{equation}
    \ket{\psi}=\sum_{n_{i}=0,1}\Bigg(\alpha_{n_{1},...,n_{N}}\bigotimes_{i=1}^{N}\ket{n_{i}}\Bigg),
\end{equation}

\noi where $\alpha_{n_{1},...,n_{N}}\in\mathbb{C}$ and
\begin{equation}
    \bigotimes_{i=1}^{N}\ket{n_{i}}=\Bigg(\bigotimes_{i=1}^{N}\big(a^{\da}_{i}\big)^{n_{i}}\Bigg)\Bigg(\bigotimes_{i=1}^{N}\ket{0}\Bigg).
\end{equation}

\subsection{Quadratic Hamiltonians}


